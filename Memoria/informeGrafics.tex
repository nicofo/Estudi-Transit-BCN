%%%%%%%%%%%%%%%%%%%%%%%%%%%%%%%%%%%%%%%%%
% University Assignment Title Page 
% LaTeX Template
% Version 1.0 (27/12/12)
%
% This template has been downloaded from:
% http://www.LaTeXTemplates.com
%
% Original author:
% WikiBooks (http://en.wikibooks.org/wiki/LaTeX/Title_Creation)
%
% License:
% CC BY-NC-SA 3.0 (http://creativecommons.org/licenses/by-nc-sa/3.0/)
% 
% Instructions for using this template:
% This title page is capable of being compiled as is. This is not useful for 
% including it in another document. To do this, you have two options: 
%
% 1) Copy/paste everything between \begin{document} and \end{document} 
% starting at \begin{titlepage} and paste this into another LaTeX file where you 
% want your title page.
% OR
% 2) Remove everything outside the \begin{titlepage} and \end{titlepage} and 
% move this file to the same directory as the LaTeX file you wish to add it to. 
% Then add \input{./title_page_1.tex} to your LaTeX file where you want your
% title page.
%
%%%%%%%%%%%%%%%%%%%%%%%%%%%%%%%%%%%%%%%%%

%----------------------------------------------------------------------------------------
%	PACKAGES AND OTHER DOCUMENT CONFIGURATIONS
%----------------------------------------------------------------------------------------

\documentclass[12pt]{article}
\usepackage[spanish]{babel}
\usepackage[utf8]{inputenc}
\usepackage{float}
\usepackage{graphicx}
\usepackage{wrapfig}
\usepackage{lscape}
\usepackage{rotating}
\usepackage{epstopdf}

\begin{document}
\begin{titlepage}

\newcommand{\HRule}{\rule{\linewidth}{0.5mm}} % Defines a new command for the horizontal lines, change thickness here

\center % Center everything on the page
 
%----------------------------------------------------------------------------------------
%	HEADING SECTIONS
%----------------------------------------------------------------------------------------

\textsc{\LARGE Universitat de Barcelona}\\[1.5cm] % Name of your university/college
\textsc{\Large Treball de Fi de Grau}\\[0.5cm] % Major heading such as course name
\textsc{\large Estudi dels accidents de trafic a Barcelona (2010-2015)}\\[0.5cm] % Minor heading such as course title

%----------------------------------------------------------------------------------------
%	TITLE SECTION
%----------------------------------------------------------------------------------------

\HRule \\[0.4cm]
{ \huge \bfseries Práctica 2}\\[0.4cm] % Title of your document
\HRule \\[1.5cm]
 
%----------------------------------------------------------------------------------------
%	AUTHOR SECTION
%----------------------------------------------------------------------------------------



% If you don't want a supervisor, uncomment the two lines below and remove the section above
\Large \emph{Author:}\\
Nicolás Martín \textsc{Forteza Ocaña}\\[3cm] % Your name
\Large \emph{Tutor:}\\
Jordi Vitrià

%----------------------------------------------------------------------------------------
%	DATE SECTION
%----------------------------------------------------------------------------------------

{\large \today}\\[3cm] % Date, change the \today to a set date if you want to be precise

%----------------------------------------------------------------------------------------
%	LOGO SECTION
%----------------------------------------------------------------------------------------

%\includegraphics{Logo}\\[1cm] % Include a department/university logo - this will require the graphicx package
 
%----------------------------------------------------------------------------------------

\vfill % Fill the rest of the page with whitespace

\end{titlepage}

\tableofcontents

\newpage

\section{Introducción}
En esta practica se nos pide crear una escena renderizada utilizando el método de \emph{RayTracing}.

Para ello tendremos que crear los objetos(esfera, plano y triangulo), hacer las transformaciones de las coordenadas de la pantalla hasta las del mundo, calcular luces y sombras y, por último, el \emph{RayTracing}, que  nos permitirá generar reflexiones i transparencias.

\section{Transformación de las coordenadas}
Para poder crear un rayo de cada píxel de la pantalla a las coordenadas del mundo, primero, hay que crear unas matrices de transformación(gracias a \textbf{lookAt}() i \textbf{perspective}()) 	para pasar de unas coordenadas a otras. Después en el método Render habrá que ir coordenada de pantalla por coordenada transformándola a las coordenadas de mundo. Gracias a esto tendremos ya los puntos para generar nuestras rectas. 
Para encontrar las matrices ha sido necesario calcular el ángulo de apertura a partir del \textit{aspect ratio}.

\section{Construcción de los objetos}
Hemos creado tres objetos: la esfera, el plano y el triangulo.

\begin{itemize}


\item \textbf{Esfera}: Hemos definido la esfera por: un punto centra, un radio y su material.

Para hacer la intersección debíamos calcular la resolución de una ecuación de segundo grado(ref: https://en.wikipedia.org/wiki/Line%E2%80%93sphere_intersection)

\item \textbf{Plano}: Hemos definido la esfera por: un punto, una normal y su material.

Para hacer la intersección debíamos calcular la resolución de una ecuación  bastante simple.	(ref: https://en.wikipedia.org/wiki/Line%E2%80%93plane_intersection)

\item \textbf{Triangulo}: Hemos definido la esfera por: tres puntos, una normal y su material.

Para hacer la intersección calculábamos la intersección del plano, luego comprobamos si estaba dentro utilizando la propiedad de que las suma de las áreas de los sub-triangulos al punto, es la misma, que el área total.

\end{itemize}



\section{Iluminación y sombras}
Calculamos el color de cada punto recorriendo todas las luces y calculando su intensidad sobre este.
Para el calculo de la iluminación, se ha reutilizado el código de Bling-Phong de la practica anterior.
Para el calculo de las sombras, hemos hecho otro método de intersección(CheckIntersectionLight), que nos retorna un valor entre el 0 y el 1 que nos dice el grado de sobra que hay. Esto se hace así para el caso de las transparencias, que generan una pequeña sombra.

\section{Reflexiones}
Para poder llevar a cabo este apartado tuvimos que hacer el método CastRay recursivo.
Esto nos ha permitido poder calcular el ultimo color antes que los demás, y así poder relegarlo en los anteriores. Solo hay un caso en que no calculamos las reflexiones, cuando el rayo comienza dentro de una esfera, esto lo hemos hecho para que en el caso de una transparencia no refleje cuando el rayo este dentro.

Finalizamos la recursión en dos casos, cuando ha pasado el máximo de Bounces o cuando el rayo va hacia el fondo.
\section{Transparencias}
Para implementar las transparencias añadimos nuevos atributos al materia: float coefRefrac, float fah, vec3 transparent y bool isTransp.

Hacemos otra llamada recursiva pero en otra dirección diferente a la del reflexión. La dirección esta definida según los materiales de entrada y salida. Para ello hacemos unas comprobaciones con la esfera de si el rayo de transparencia esta entrando o saliendo, así podemos saber cual es el material de entrada y cual el de salida. Con los planos y los triángulos no tenemos este problema ya que no tienen volumen.
Luego le sumamos la transparencia el la intensidad resultante de Bling-Phong.

\section{Pruebas}
Test antes de añadir la iluminación.
\begin{figure}[H]
\centering
\includegraphics[scale=0.4]{inicialgrafics.png}
\caption{Sin iluminación.}\label{visina8}
\end{figure}

Test con iluminación.

\begin{figure}[H]
\centering
\includegraphics[scale=0.4]{sinSombra.png}
\caption{Con iluminación.}\label{visina8}
\end{figure}

\newpage
Test con sombras.

\begin{figure}[H]
\centering
\includegraphics[scale=0.4]{sombra.png}
\caption{Con sombras.}\label{visina8}
\end{figure}

Test con triangulo(extensión)

\begin{figure}[H]
\centering
\includegraphics[scale=0.4]{triangulo.png}
\caption{Triangulo.}\label{visina8}
\end{figure}

\newpage
Test con reflexiones.

\begin{figure}[H]
\centering
\includegraphics[scale=0.4]{reflex.png}
\caption{Con reflexiones.}\label{visina8}
\end{figure}

Test de diferentes transparencias de una bola.

\begin{figure}[H]
\centering
\includegraphics[scale=0.4]{trans1.png}
\caption{Transparencia con material 1.0.}\label{visina8}
\end{figure}

\begin{figure}[H]
\centering
\includegraphics[scale=0.4]{trans2.png}
\caption{Transparencia con material 1.33.}\label{visina8}
\end{figure}

\begin{figure}[H]
\centering
\includegraphics[scale=0.4]{trans3.png}
\caption{Transparencia con material 2.0.}\label{visina8}
\end{figure}

\newpage
Escena final.
\begin{figure}[H]
\centering
\includegraphics[scale=0.4]{escena.png}
\end{figure}


\section{Conclusión}

Gracias a esta práctica hemos consolidado todos los conocimientos que se nos quería enseñar durante las horas de teoría. No es lo mismo como se plantean los cocimientos en la aula que en la sesión de practicas, se puede hacer muy cuesta arriba, pero al final los resultados son satisfactorios.
\end{document}